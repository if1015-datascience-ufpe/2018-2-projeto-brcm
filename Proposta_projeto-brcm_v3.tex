\documentclass{article}
\usepackage[utf8]{inputenc}

\title{Análise da situação financeira do município de Recife}

\author{
  Rios, Breno\\
  \texttt{brr@cin.ufpe.br}
  \and
  Moreira, Cinthya\\
  \texttt{cml2@cin.ufpe.br}
}

\date{Novembro 2018}

\begin{document}

\maketitle
\setlength{\parindent}{4em}
\setlength{\parskip}{1em}
\renewcommand{\baselinestretch}{2.0}

\section{Proposta}
\text{
Realizar uma análise financeira dos município de Recife, com o objetivo de identificar como é gasto o dinheiro arrecadado e se este gasto está dentro dos limites da LOA. Serão feitas também comparações com as receitas para saber se a gestão está ocorrendo de maneira responsável.
}

\section{Etapas}

 \begin{enumerate}
   \item Conversão dos dados para o formato adequado de manipulação e transformação das bases que necessitam em dataframes.
   \item Pré-processamento dos dados: tratamento de dados que estão faltando, remoção de possíveis atributos não utilizados e ajuste de tipos.
   \item Ajuste de atributos que estão identificados por códigos e convertê-los para um identificador com melhor significado.
   \item Criar gráficos mostrando a evolução dos gastos dde acordo com sua função.
   \item Cruzar dados anteriores com os dados financeiros com o da LOA (LEI Nº 18.281 /2016)
   \item Estruturação da estória dos dados a serem apresentados, organizando as análises feitas em uma ordem relevante para a extração de conclusões relevantes.
 \end{enumerate}

\section{Datasets}
\begin{itemize}
    \item Dados sobre as receitas e despesas da cidade de Recife:
        \begin{itemize}
            \item http://dados.recife.pe.gov.br/dataset?q=finan%C3%A7as
        \end{itemize}

    \item LOA 2017:
        \begin{itemize}
            \item https://leismunicipais.com.br/PE/RECIFE/LEI-18281-2016-RECIFE-PE.pdf
        \end{itemize}  
 \end{itemize}

\end{document}
