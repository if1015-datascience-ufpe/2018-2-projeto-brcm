\documentclass{article}
\usepackage[utf8]{inputenc}

\title{Análise da situação financeira dos municípios do estado de Pernambuco}

\author{
  Rios, Breno\\
  \texttt{brr@cin.ufpe.br}
  \and
  Moreira, Cinthya\\
  \texttt{cml2@cin.ufpe.br}
}

\date{Novembro 2018}

\begin{document}

\maketitle
\setlength{\parindent}{4em}
\setlength{\parskip}{1em}
\renewcommand{\baselinestretch}{2.0}

\section{Proposta}
\text{
Realizar uma análise financeira dos municípios do Estado de Pernambuco, com o objetivo de identificar os municípios que não conseguem se sustentar com verbas próprias, ou ainda aqueles que não justificam sua existência como município independente, de acordo com as leis vigentes para a criação de novos municípios. Para isso será utilizada as bases dados disponíveis para o ano de 2017 tanto no portal de dados abertos do Tribunal de Contas de Pernambuco quanto no Portal da Transparência de Pernambuco, que contém informações sobre as receitas e despesas de cada município do estado. Além dessas, para complementar, utilizaremos bases sobre indicadores demográficos também contidos nesses portais e no portal da Base de Dados do Estado para analisar como a situação financeira tem impactado nessas áreas.

Cada registro além de informações para identificação, contém o código da unidade gestora que gerou aquela despesa ou receita, a categoria, a origem, espécie e descrição, e também o valor daquele lançamento. Como cada registro tem a identificação da Unidade Gestora que o gerou, precisamos de uma tabela de dados adicional disponível também no portal do Tribunal de Contas de Pernambuco que mostra qual o nome da unidade gestora (pode ser uma prefeitura ou assembléia legislativa de algum município) e também a qual município ela pertence, tornando possível além de conseguir agrupar as receitas ou gastos por município, também averiguar mais detalhadamente os gastos dentro dele observando as Unidades Gestoras e seus gastos. 

Para a parte de visualização, usaremos os dados encontrados no portal GMapas que contém os limites geográficos de cada município pernambucano, o que será muito útil na construção de, por exemplo, mapas de calor.
}

\section{Etapas}

 \begin{enumerate}
   \item Conversão dos dados para o formato adequado de manipulação (.kmz e .xml) e transformação das bases que necessitam em dataframes.
   \item Pré-processamento dos dados: tratamento de dados que estão faltando, remoção de possíveis atributos não utilizados e ajuste de tipos.
   \item Ajuste de atributos que estão identificados por códigos e convertê-los para um identificador com melhor significado (ex.: transformar o código da Unidade Gestora para o nome dessa unidade e também adicionar o Município a que pertence).
   \item Criar gráficos mostrando a evolução dos gastos dos municípios ao longo dos anos, criação de mapa de calor do estado de acordo com a situação financeira de cada município.
   \item Cruzar dados anteriores com os dados demográficos (população, escolarização, IDH, mortalidade) de cada município.
   \item Criar um modelo de predição que dada a situação financeira do município estime alguns indicadores sociais como o IDHM.
   \item Estruturação da estória dos dados a serem apresentados, organizando as análises feitas em uma ordem relevante para a extração de conclusões relevantes.
 \end{enumerate}

\section{Datasets}
\begin{itemize}
    \item Dados sobre as receitas e despesas de cada ano divididos por Unidades Gestoras:
        \begin{itemize}
            \item https://www.tce.pe.gov.br/internet/index.php/dados-abertos/bases-de-dados-completas 
        \end{itemize}

    \item Arquivo .kmz (pode ser convertido para .csv) com os limites geográficos de cada município:
        \begin{itemize}
            \item http://www.gmapas.com/poligonos-ibge/poligonos-municipios-ibge-pernambuco
        \end{itemize}

    \item Dados sobre a posição geográfica das sedes dos municípios:
        \begin{itemize}
            \item http://www.bde.pe.gov.br/visualizacao/Visualizacao_formato2.aspx?codFormatacao=703&CodInformacao=280&Cod=1 
        \end{itemize}

    \item Dados demográficos de cada município de Pernambuco:
        \begin{itemize}
            \item https://www.ibge.gov.br/estatisticas-novoportal/por-cidade-estado-estatisticas.html?t=destaques&c=26 
            \item http://www.bde.pe.gov.br/site/ConteudoRestrito2.aspx?codGrupoMenu=84&codPermissao=5 
        \end{itemize}

    \item Dados sobre as despesas do Estado de Pernambuco (contém os repasses feitos aos municípios):
        \begin{itemize}
            \item http://web.transparencia.pe.gov.br/dados-abertos/   
            (*não há link específico para esse dataset, mas é especificamente o dataset sobre Despesas) 
        \end{itemize}
 \end{itemize}

\end{document}
